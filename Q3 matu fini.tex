% !TeX spellcheck = fr_FR
\documentclass[12pt,a4paper]{report}
\usepackage{style/preambule_college}
\usepackage{style/preambule_personnalisation}
\usepackage{geometry}
\usepackage{amsmath}
\usepackage{tikz}
\usepackage{xcolor}
\usepackage{pgfplots}
\chapterFormat


\begin{document}
	
	\chapter[Analyse]{Logarithme naturel}
	\section*{Définition}
	La fonction est définie de $\mathbb{R}^*_{+} \rightarrow \mathbb{R}$.
	
	Elle est une bijection entre ses 2 ensembles.
	La fonction ln
	\section*{Graphe}
	\begin{tikzpicture}
	\begin{axis}[
		xscale=1,yscale=1,
		xmin=-1, xmax=10,
		ymin=-4, ymax=4,
		samples=1000,
		axis lines=center,
		]
		\addplot+[domain=0:10,mark=none,blue] {ln(x)};
	\end{axis}
	\end{tikzpicture}
	\section*{Dérivée de $\ln(x)$}
	\begin{align*}
	(\ln(x))' &= \lim\limits_{h \rightarrow 0} \dfrac{\ln(x+h) - \ln(x)}{h} \\
	&= \lim\limits_{h \rightarrow 0} \dfrac{\ln\left(\frac{x+h}{x}\right)}{h} \\
	&= \lim\limits_{h \rightarrow 0} \dfrac{\ln\left(1+\frac{h}{x}\right)}{h} \\
	&= \lim\limits_{h \rightarrow 0} \frac{\color{blue}x}{\color{blue}x} \cdot \dfrac{\ln\left(1+\frac{h}{x}\right)}{h} \\
	&= \lim\limits_{h \rightarrow 0} \frac{1}{\color{blue}x} \cdot \ln\left(1+\frac{h}{x}\right)^{\frac{\color{blue}x}{h}}
	\end{align*}
	Or si $h \rightarrow 0$ alors $ \frac{x}{h} \rightarrow +\infty$ et $ \frac{h}{x} \rightarrow 0$. On en revient donc à la définition du nombre $e$. Il ne reste donc plus que $\frac{1}{x}$.
	\begin{align*}
	&= \dfrac{1}{\color{blue}x}\cdot\ln(e) \\
	&= \dfrac{1}{x}
	\end{align*}
\end{document}