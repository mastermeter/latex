% !TeX spellcheck = fr_FR
\documentclass[12pt,a4paper]{report}
\usepackage{style/preambule_college}
\usepackage{style/preambule_personnalisation}
\usepackage{geometry}
\usepackage{amsmath}
\chapterFormat


\begin{document}
	\chapter[Géométrie dans l'espace]{Sphère:définition, équation cartésienne. Equation d'un plan tangent à une sphère}
	\section*{Définition de la sphère}
	Une première définition que l'on peut donner de la sphère est: tous les point à équidistance d'un seul et point (que l'on va appeler C le centre du cercle). La sphère se définie donc par sont centre $C(x,y,z)$ et un rayon r. La lettre pour signifier une sphère est $\Sigma$.
	\vspace{10cm}
	
	Soit P($x_0;y_0;z_0$) un point de la sphère. $\delta(C;P)=r \Leftrightarrow \| \overrightarrow{CP} \|=r $.
	En utilisant le théorème de Pythagore on peut trouver l'égalité: \[ \sqrt{(x-x_0)^2+(y-y_0)^2+(z-z_0)^2}=r \] 
	
	Puis en élevant au carré cette expression: \[ (x-x_0)^2+(y-y_0)^2+(z-z_0)^2=r^2 \]
	On obtient l'équation cartésienne de la sphère.
	\pagebreak
	
	\section*{Plan tangent à la sphère}
	\vspace{10cm}
	Soit une sphère $\Sigma$ d'équation : $(x-2)^2+(y+1)^2+(z-3)^2=30$. Soit le plan $\pi$ passant par T $(1,4,5)$ et P de telle manière à ce que $\overrightarrow{CT}\cdot\overrightarrow{TP}$. Il sera question ici de trouver ce point P. 
	\[CT=\left( \begin{array}{c}
	-1 \\
	5 \\
	2
	\end{array} \right) \hspace{1cm} TP=\left( \begin{array}{c}
						x-1 \\
						y-4 \\
						z-5
						\end{array} \right)\]
	
	Ce qui donne: $-(x-1)+5(y-4)+2(z-5)=0 \Leftrightarrow x-5y-2z+29=0$
	Cette démarche pour trouver le plan tangent nous a fait chercher un point P qui n'appartient à la sphère pour et qui donne un produit scalaire nul, donc que les deux vecteurs sont perpendiculaires. 
	
\end{document}