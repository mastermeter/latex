% !TeX spellcheck = fr_FR
\documentclass[12pt,a4paper]{report}
\usepackage{style/preambule_college}
\usepackage{style/preambule_personnalisation}
\usepackage{geometry}
\usepackage{amsmath}
\chapterFormat
\begin{document}

	\chapter[Géométrie dans l'espace]{Equation d'une droite. Equation d'un plan. Position relative d'un plan et d'une droite}
	\section*{Droite}
	\vspace*{10cm}
	\subsection*{Equation}
	\subsubsection*{Géométrique}
	$\overrightarrow{OP}=\overrightarrow{OA}+\lambda \overrightarrow{d}$
	\subsubsection*{Paramétrique}
	$\left\{
	\begin{array}{l}
	x=a_1 + \lambda d_1 \\
	y=a_2 + \lambda d_2 \\
	z=a_3 + \lambda d_3
	\end{array}
	\right.$
	\subsubsection{Cartésienne}
	$\dfrac{x-a_1}{d_1}=\dfrac{y-a_2}{d_2}=\dfrac{z-a_3}{d_3}$
	\pagebreak
	\section*{Plan}
	\vspace{10cm}
	\subsection*{Equation}
	\subsubsection*{Géométrique}
	$\overrightarrow{OP}=\overrightarrow{OA} + \lambda \overrightarrow{u} + \mu \overrightarrow{v}$,$\lambda,\mu \in \mathbb{R}$
	\subsubsection*{Paramétrique}
	$\left\{
	\begin{array}{l}
	x=a_1 + \lambda u_1 + \mu v_1 \\
	y=a_2 + \lambda u_2 + \mu v_2 \\
	z=a_3 + \lambda u_3 + \mu v_3
	\end{array}
	\right.$
	\subsubsection*{Cartésienne}
	$ ax+by+cz+d=0 $
	\pagebreak
	\section*{Position relative d'une droite et d'un plan}
	Soit un plan $\pi:ax+by+cz+d=0$ et la droite $d=\left\{
	\begin{array}{l}
		x=a_1 + kd_1 \\
		y=a_2 + kd_2 \\
		z=a_3 + kd_3
	\end{array}
	\right.$
	
	3 cas possibles: sécant, parallèles, confondus
	
	\subsection*{Substitution de l'équation de la droite dans le plan}
	En substituant $x$ par $ a_1 + kd_1 $ puis $y$ par $a_2 +  kd_2$ puis $z$ par $a_3 + kd_3$
	et en simplifiant l'expression pour sortir une valeur pour k: si $k=$ cst alors la droite et le plan sont sécants, si le résultat est impossible (plus de k et on arrive à une égalité $5=2$) alors la droite est strictement parallèle au plan et si le résultat admet une infinité de possibilité ($5=5$) alors la droite est confondue au plan.
	
	\subsection*{Sécant}
	Lorsque la droite coupe le plan, il est intéressant de voir quel est ce point (par le calcul fait précédemment) et quel est l'angles entre le plan et la droite.
	Pour cela, il faut nécessairement passer par le calcul d'angle entre 2 droites. On utilise alors le vecteur normal au plan $\overrightarrow{n}=
	\left( \begin{array}{c}
		a \\
		b \\
		c
	\end{array} \right)$
	et on utilise le calcul de l'angle entre 2 vecteurs directeurs $\cos(\phi)= \dfrac{\mid \overrightarrow{d} \cdot \overrightarrow{n} \mid}{\|\overrightarrow{d}\|\|\overrightarrow{n}\|}$
	Cependant, ce calcul nous donnes l'angles entre les droite $n$ et $d$ il faut prendre le complémentaire de cet angles et donc l'angles entre la droite et le plan est: $\phi= 90°- \arccos \left(\dfrac{\mid \overrightarrow{d} \cdot \overrightarrow{n} \mid}{\|\overrightarrow{d}\|\|\overrightarrow{n}\|}\right)$
	
	\subsection*{Strictement parallèle}
	Dans le cas où la droite est strictement parallèle au plan, donc que le calcul précédent a donné un valeur impossible. On peut regarder la distance de cette droite au plan. En utilisant le même calcul pour la distance d'un point à un plan. $\delta(d;\pi)\Leftrightarrow\delta(P;\pi)$ pour $P \in d$ \[ \delta(P;\pi) = \dfrac{\mid \overrightarrow{AP}\cdot \overrightarrow{n}\mid}{\|\overrightarrow{n}\|} = \dfrac{\mid ax_P + by_P + cz_P + d \mid}{\sqrt{a^2+b^2+c^2}} \] $A \in \pi$ 
	\subsection*{Confondue}
	Il n'y a rien d'intéressant à voir dans ce cas.
	
\end{document}
